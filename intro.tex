\section{Introduction}
\label{sec:intro}
\fix{Dropped Jim's initial email here.}
The Cray XK7 Titan is rapidly nearing its decommissioning date for
DOE/ASCR. The last production runs will be in July of this year, and
then we will stop services, clean up a few things and turn it off. 
 
This system was \#1 in the world for a very long time, and has remained
a critically important computer system. It defied scale with 18,688
individual compute nodes, and delivered tens of billions of computing
hours to DOE mission-critical programs for nearly 7 years. 
 
This was quite an interesting machine from a reliability perspective.
We were forced to execute three very significant rework cycles, two on
the mechanical assembly affecting the PCIe connector from the GPU
daughtercard to the motherboard, and more recently, a replacement of
about 11,000 GPU assemblies because of a failing resistor on a printed
circuit board. 
 
We have conducted a good bit of statistical analysis on the failure
rates of these GPU assemblies, but I want to propose a different topic
that I have not seen for other/similar systems.  I want to complete
the statistical analysis of the failure rates as we near end of life,
looking for the opposite side of the “bathtub curve”.  There is
specific decay associated with the electronic components where, as
these products age, they will eventually fail thresholds for guard
band margin and other things. In a lot of cases, they will just fail
as well. 
 
This machine is quite large, was very important to many people, and I
think this analysis could be very intriguing. 
 
I am looking for help, very specifically with the statistics portion
of this, i.e. what do the failure rates for specific components look
like? What distributions do those failure rates follow? How could
appropriate models extrapolate failure rates beyond the end of service
date (the dotted lines…) ? 
 
I have submitted an abstract to a Cray-focused user group and workshop
on this topic, and it has been accepted.  I would like to execute this
statistical analysis and generate a short presentation (30 minutes)
and a short paper (4-8 pages is a likely target).  Presentation is
targeted for May 7-9 (TBD). An initial/working version of the paper
would be due earlier. If we miss that target (April 12), no big
deal. We can finish on a similar schedule as the presentation.  I have
all of the raw data for the component failures, and need the
statistical help to complete the analysis. 
 
Because there is a huge swing in the failure rates of the GPUS from
2016/17 when the resistor problem escalated rapidly out of control, it
is statistically interesting, where we inject fresh material in to an
existing machine. We’d likely want to categorically separate new GPU
assemblies from old and could track the early life failures from those
new 11,000 parts versus the continued failure of the remaining 6000. 
 
George, Christian-  any interest in this? Barney gave it a thumbs
up. I think this is a timely and very interesting topic. 

\fix{Need a description of Titan GPU events timeline. Jim or Don?}
