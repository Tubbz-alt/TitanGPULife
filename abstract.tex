% !TEX root = main.tex
%%%%%%%%%%%%%%%%%%%%%%%%%%%%%%%%%%%%%%%%%%%%%%%%%%%%%%%%%%%%%%%%%%%%%%%%%%%%%%%%
% Abstract
%%%%%%%%%%%%%%%%%%%%%%%%%%%%%%%%%%%%%%%%%%%%%%%%%%%%%%%%%%%%%%%%%%%%%%%%%%%%%%%%

The Cray XK7 Titan was the top supercomputer system in the world for a
very long time and remained critically important throughout its nearly
seven year life. It was also a very interesting machine from a
reliability viewpoint as most of its power came from 18,688 GPUs whose
operation was forced to execute three very significant rework cycles,
two on the GPU mechanical assembly and one on the GPU circuitboards. We
write about the last rework cycle and a reliability analysis of over
100,000 operation years in the GPU lifetimes, which correspond to
Titan's 6 year long productive period after an initial break-in
period. Using time between failures analysis and statistical survival
analysis techniques, we find that GPU reliability is dependent on heat
dissipation to an extent that strongly correlates with detailed
nuances of the system cooling architecture. In addition to describing
some of the system history, the data collection, data cleaning, and
our analysis of the data, we provide reliability recommendations for
designing future state of the art supercomputing systems and their
operation. We make the data and our analysis codes publicly available.
